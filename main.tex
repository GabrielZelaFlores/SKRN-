\documentclass[a4paper,12pt]{article}
\usepackage[utf8]{inputenc}
\usepackage{graphicx}
\usepackage{listings}
\usepackage{xcolor}
\usepackage{hyperref}

% Configuración de código en LaTeX
\lstset{
    language=Python,
    basicstyle=\ttfamily\footnotesize,
    keywordstyle=\color{blue},
    commentstyle=\color{gray},
    stringstyle=\color{red},
    breaklines=true,
    frame=single,
    showstringspaces=false
}

\title{\textbf{Especificación Léxica del Lenguaje SKRN}}
\author{
    Hugo Alonso Youzzueff Diaz Chavez\\
    Juan José Huamaní Vásquez\\
    Melvin Jarred Yabar Carazas\\
    Gabriel Frank Krisna Zela Flores
}
\date{20 de Marzo de 2025}

\begin{document}

\maketitle
\begin{center}
    \textbf{Universidad La Salle - Escuela Profesional de Ingeniería de Software}\\
    \textbf{Curso: Compiladores}\\
    \textbf{Docente: Vicente Enrique Machaca Arceda}
\end{center}

\newpage

\section{Introducción}
El lenguaje \textbf{SKRN} es un lenguaje de programación diseñado para facilitar el aprendizaje de programación utilizando palabras clave en español escritas al revés. En este informe, se detalla la implementación del analizador léxico para este lenguaje.

\section{Marco Teórico}
El analizador léxico es la primera fase de un compilador. Su tarea principal es transformar el código fuente en una secuencia de tokens reconocibles por el analizador sintáctico. Para este proyecto, se utiliza la librería \textbf{PLY (Python Lex-Yacc)}.

\section{Definición de Tokens}
La siguiente tabla describe los tokens reconocidos por el lenguaje SKRN:

\begin{center}
\begin{tabular}{|c|c|c|}
    \hline
    \textbf{Token} & \textbf{Expresión Regular} & \textbf{Descripción} \\
    \hline
    ID & [a-zA-Z][a-zA-Z0-9]* & Identificadores \\
    INTEGER & [0-9]+ & Números enteros \\
    FLOATING & [0-9]+\.[0-9]+([eE][+-]?[0-9]+)? & Números flotantes \\
    ASSIGN & = & Operador de asignación \\
    PLUS & + & Operador de suma \\
    MINUS & - & Operador de resta \\
    TIMES & * & Operador de multiplicación \\
    DIVIDE & / & Operador de división \\
    LPAREN & ( & Paréntesis izquierdo \\
    RPAREN & ) & Paréntesis derecho \\
    LBRACE & \{ & Llave izquierda \\
    RBRACE & \} & Llave derecha \\
    IF & fi & Condicional if \\
    ELSE & esle & Condicional else \\
    RETURN & nruter & Retorno de función \\
    \hline
\end{tabular}
\end{center}

\section{Implementación del Analizador Léxico}
A continuación, se presenta el código en Python para la implementación del analizador léxico utilizando PLY:

\begin{lstlisting}
import ply.lex as lex

# Lista de nombres de los tokens (según tu tabla de tokens)
tokens = [
    'ID', 'INTEGER', 'FLOATING',
    'ASSIGN', 'PLUS', 'MINUS', 'TIMES', 'DIVIDE', 'MODULO',
    'INCREMENT', 'DECREMENT', 'EQUAL', 'NOT_EQUAL',
    'LESS', 'GREATER', 'LESS_EQUAL', 'GREATER_EQUAL',
    'LOGICAL_AND', 'LOGICAL_OR', 'LOGICAL_NOT',
    'PLUS_ASSIGN', 'MINUS_ASSIGN', 'TIMES_ASSIGN', 'DIV_ASSIGN',
    'TERNARY_Q', 'TERNARY_C', 'MEMBER_ACCESS', 'POINTER_ACCESS',
    'LPAREN', 'RPAREN', 'LBRACE', 'RBRACE', 'LBRACKET', 'RBRACKET',
    'SEMICOLON', 'COMMA', 'STRING', 'CHARACTER'
]

# Palabras reservadas (reversed)
reserved = {
    'tni': 'type_int',
    'taolf': 'type_float',
    'fi': 'IF',
    'esle': 'ELSE',
    'nruter': 'RETURN'
}

tokens += list(reserved.values())

# Expresiones regulares de operadores y delimitadores
t_ASSIGN = r'='
t_PLUS = r'\+'
t_MINUS = r'-'
t_TIMES = r'\*'
t_DIVIDE = r'/'
t_MODULO = r'%'
t_EQUAL = r'=='
t_NOT_EQUAL = r'!='
t_LPAREN = r'\('
t_RPAREN = r'\)'
t_LBRACE = r'\{'
t_RBRACE = r'\}'

lexer = lex.lex()

data = """
tni niam() {
    tuoc << "Hola, Mundo!"
    nruter
}
"""
lexer.input(data)

while True:
    tok = lexer.token()
    if not tok:
        break
    print(tok)
\end{lstlisting}

\section{Tokens Generados}
Ejecutando el código anterior, se obtiene la siguiente salida de tokens:

\begin{verbatim}
Token(type='type_int', lexeme='tni', line=1, column=1)
Token(type='ID', lexeme='niam', line=1, column=5)
Token(type='LPAREN', lexeme='(', line=1, column=9)
Token(type='RPAREN', lexeme=')', line=1, column=10)
Token(type='LBRACE', lexeme='{', line=2, column=1)
Token(type='ID', lexeme='tuoc', line=3, column=5)
Token(type='STRING', lexeme='"Hola, Mundo!"', line=3, column=10)
Token(type='RETURN', lexeme='nruter', line=4, column=5)
Token(type='RBRACE', lexeme='}', line=5, column=1)
\end{verbatim}

\section{Conclusión}
La implementación del analizador léxico para el lenguaje SKRN demuestra la viabilidad de utilizar palabras clave en español invertidas para el aprendizaje de programación. La combinación de PLY y expresiones regulares permite reconocer tokens de manera eficiente.

\end{document}